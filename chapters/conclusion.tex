%! TEX root = /home/duy/TUB/Thesis/latex-source/DiplomarbeitLaTex.tex

\chapter{Conclusion and Future Work \label{cha:conclusion}}

In conclusion, our experiments show that Depth Data is contributive and adding depth data
from our Depth-GAN is beneficial to training an Object Classifier to different extents. 
It is worth noticing that, because of a drawback of our train-test split explained in
section \ref{sec:train_test_split}, our \acrshort{gan} creating different proportions
are not equal. This is clearly demonstrated in section \ref{sub:depth_gan}. However,
despite the disadvantages it has, the \acrshort{gan} trained by 10\% of the data still
performs greatly by improving the classification results significantly, comparing to the
classifier trained on only 10\% of the original data. The same is true for the
\acrshort{gan} trained with 25\% of the original data.

Therefore, the Depth-GAN is particularly useful when our "real" training data is not
complete and the synthesized data is added in a large proportion (the 10-90 case). The use
case of it is when we have a large training set but only a small portion of images have
depth maps included, or when there is no depth maps at all, then it is a good idea to
synthesize them.

In the way around, if we consider the baseline classifier as a quantitative tool to
evaluate \acrshort{gan} data, then our Depth-GAN is showing its potentials. The results
are good in both ways: to human eyes, as demonstrated in Figure~\ref{fig:gan_depth}, and
in quantitative metrics, shown in the accuracies plot \ref{fig:eitel_accuracies} and
T-Test results \ref{tab:t_test}.

The Pose-GAN, in another way, only does good jobs on some particular objects. It is
understandable as its task is much more complicated than the Depth-GAN and other
\acrshort{pix} applications. It is not a simple Image-to-Image translation task, but
requires a good understanding about the 3D space. Although we have provided the depth map
(as the 4th dimension), the Pose-GAN still does not utilize this information well and
cannot capture the full 3D space distribution of the objects. A reason could be that this
setting (4-dimensional image with 4 channels R,G,B, and D) is still not optimal. A
reasonable future work is to try another depth representation when training the Pose-GAN.

While training the baseline classifier with both the original and the synthesized data,,
we observe a behavior that the network learns much more from the RGB channels than the
Depth one. We tried to regularize the RGB channels in different ways, such as adding
Gaussian Noise to the RGB frame and Dropouts, but cannot change the Network's learning
behavior. Similar to the point above, using another representation of Depth, instead of
colorizing and treating it as a second RGB frame, could be the answer and could improve
the classification results even further.

Furthermore, as the Discriminator of \acrshort{gan}s is trained to perform good
classification between real data and the synthesized data from the Generator, it can also
contain layers with useful information about the dataset. Another feasible future work is
to take the weight vectors of one or a few layers before the Softmax as the
representations for other tasks, such as our baseline classification.

