\chapter{Introduction\label{cha:introduction}}
\section{Motivation\label{sec:moti}}
% intro to machine learning
\subsection{Context}
In the recent years, \acrfull{ml} has become a powerful tool for computer scientists,
business analysts and other professionals to improve their work because it can make use of
a large amount of data using probabilistic models. It is able to capture properties of
different objects and events by making observations, which is difficult to do using human
labour or traditional rule-based methods which require many manual engineering works.
\acrshort{ml} became popular thanks to the boost of data generation in the Internet era. 

%There are three basic approaches for any \acrshort{ml} problems: Supervised Learning,
%Unsupervised Learning and Reinforcement Learning. In any of the approaches, the main job
%of the learning machine is to optimize an objective function. In order to achieve that,
%the learning machine is given a set of models and pick a model which it considers to be
%the best. In a lot of settings, this work is similar to approximating a function on a
%population through a sampled set of representatives - the "training set".

% Intro to Deep Learning
%There are many different approaches to approximate a function on a dataset. For instance,
%the most simple one is to assume that the data is linearly distributed and thus just find
%the best linear function $\mathbf{w}x + b$ fitting the data, where $x$ is the data point,
%$\mathbf{w}$ the parameter vector and $b$ the bias. There are also more complicated class
%of models such as Logistics Regression and Support Vector Machine. The common point for
%all of these methods is that we have to find a good vector representation of our data,
%otherwise the learning algorithms will not work in the most efficient way. The task is called
%``Feature Engineering'', and takes in some cases most of the time of the learning task.

There are many different \acrshort{ml} methods, in which \acrfull{dl} is currently the
most exposure \acrshort{ml} one because it could already beat traditional methods in a
lot of different fields from Computer Vision to Natural Language Processing or Sound
Recognition without the need of thorough domain knowledge. It is basically a class of
functions which consists of different layers of computing, each of which is supposedly
responsible for learning a specific level of understanding about the objective problem. In
most of the cases, Deep Learning is implemented in different forms of Multi-layer Neural
Networks.

% Why learning representations?
\subsection{The Need of Good Data Representations in \acrshort{ml}}
A common point for all of the \acrshort{ml} methods is that we have to find a good vector
representation of our data, otherwise the learning algorithms will either not work or not
produce the most reliable outcomes. The task is called "Feature Engineering" as it
extracts "Features" from raw data. For instance, in text data, document features can be
vectors of tf.idf statistics of each word in the document; in image data, image features
can be vectors of corners, edges and some other special properties of the image content.
In many cases, Feature Engineering takes most of the time of a learning job.

Because of the importance of having good features, Representation Learning has become an
important research topic in \acrshort{ml}. In traditional learning methods, this is
crucial because the set of models only understand the input in vector space. In Deep
Learning, this is less heavy because of techniques such as Convolutional Neural Networks,
which take directly the raw data and learn different levels of representations in the
hidden layers. However, having a good representation helps not only simplify the learning
task dramatically but also increase the robustness of the learning engine.

% GAN
\subsection{\acrfull{gan} \label{subsec:intro_gan}}
\acrlong{gan}s \cite{gan} are Unsupervised/Semi-Supervised Deep Neural Networks where the
learning engine consists of two competing components: The Generator and the Discriminator.
As its name indicates, the generator tries to generate data from a noise distribution,
could be conditioned on some input or have no conditions; while the discriminator tries to
classify between a real data item and the synthesized one from the generator, under the
same condition. While using supervised losses as components, \acrshort{gan} is usually
referred as an Unsupervised \acrshort{ml} task because it does not try to learn a mapping
between the data and a set of labels. Instead, \acrshort{gan} tries to learn the
generalized distribution of a particular dataset and to be able to draw samples from that
distribution. The distribution that \acrshort{gan}, or particularly the Generator, learns
can be unconditional or conditional. The original idea of \acrshort{gan} is unconditional,
whereas many following-up papers use a conditional version to better train it for specific
tasks. The advantage of \acrshort{gan} comparing to some other altenatives such as
\acrfull{vae} \cite{vae} is that \acrshort{gan} tends to produce more realistic images
with less blurry boundaries, which is a shortcoming of the L2 loss \cite{gan}.  However,
because \acrshort{gan} does not minimize a standard loss function between the outputs and
the targets, evaluating it is still a challenge. In many cases, human evaluation is used.

The nature of probabilistic distribution estimation of \acrshort{gan}s motivates
researchers to utilize it for extracting data representation. There are different ideas on
using \acrshort{gan} to improve an \acrshort{ml} task. For instance, the Generator, which
is supposed to capture a probabilistic distribution, can draw new samples from its
knowledge of the data, which could enriches the training and evaluation sets of a
\acrshort{ml} task. This is potential if the \acrshort{gan} is trained well and the
Generator can capture the data distribution.

\section{Objective \& Scope \label{sec:objective}}
The main focus of this Master Thesis is training and evaluating of \acrfull{gan} on RGB-D
data, particularly the Washington RGB-D Dataset \cite{washington_rgbd}. We would like to
find out if \acrshort{gan} is able to generate good object images, particularly useful
images as training items for an Object Recognition task. In the effort to evaluate if
\acrshort{gan} is able to help improving classification performance
or not, the Thesis demonstrates different ways of training a \acrshort{gan}
to learn from RGB and Depth data, and testing the \acrshort{gan} results on another Deep
Network for Object Recognition, which we call the ``Baseline Classifier''.

The thesis focuses on evaluating the performance of a trained \acrshort{gan}'s outputs on
training a standard \acrshort{ml} task. Particularly, we aim to train \acrshort{gan}s to
learn distributions from the Washington Object RGB-D Dataset \cite{washington_rgbd}, then
evaluate the effect that the \acrshort{gan} data have in Eitel et al. \cite{eitel}. In all
of the experiments, a conditional version of \acrshort{gan} is used. The original
unconditional \acrshort{gan} is not in the scope of this thesis.  

\subsection{Using a \acrshort{gan} to learn representations from Washington RGBD Dataset} 
In this thesis, two different settings of \acrshort{gan} are experimented. 

In the first model, the objective of the Generator is to generate a good depth map,
conditioned on an RGB frame of an object. Basically, it is the transformation of an object
from a 2D space to a 3D space. The task of the Discriminator, as usual, is to classify a
real depth image from a depth image produced by the generator.

In the second model, the objective of the Generator is to generate an RGB frame in
another pose of an item, conditioned on both the RGB and the Depth map. Briefly speaking,
this Generator tries to ``rotate'' the object by an angle. The Discriminator, as usual,
tries to classify the real RGB-D pair from the ``fake'' pair.

\subsection{Training a deep neural network to do object classification}
In this Thesis, we use the work from Eitel et al. \cite{eitel} because of several
reasons. First, it is relevant to our scope; the authors train an Object Recognition
Network using Transfer Learning from CaffeNet, one of the first reference implementation
of AlexNet \cite{alexnet}. Second, its
evaluation is based on same dataset (Washington RGB-D) and in the manner that fits us
(learning from both RGB and Depth data). Finally, it achieves high accuracies (more than
91\% when using both RGB and Depth data).

\subsection{Comparing classification performances on original data and data from
	\acrshort{gan} in different scales}

We perform the evaluation of a \acrshort{gan} in multiple steps. First, we construct our
"Baseline Classifier" by reproducing the Eitel et al. \cite{eitel} results. Second, we
substitute parts of the training data by the generated data from the corresponding
\acrshort{gan}, re-train the Baseline Classifier accordingly and compare the accuracies.
Finally, we perform some noise injection experiments to the Depth and RGB components
respectively to better understand how much those components contribute to the
classification results.

\section{Outline\label{sec:outline}}

This section gives a brief outline of the content of the following chapters in this
thesis.
\\
\\
\textbf{Chapter \ref{cha:relatedwork}} describes the relevant researches and the
foundation works which are used in this thesis. It also gives an introduction to the
background needed to follow the next chapter.
\\
\\
\textbf{Chapter \ref{cha:methodology}} discusses how we train the models and some relevant
theoretical reasoning. The evaluation strategy and the train-test split are also
thoroughly discussed.
\\
\\
\textbf{Chapter \ref{cha:model}} describes the specs and architectures of the Networks we
use together with some relevant hyper-parameters.
\\
\\
\textbf{Chapter \ref{cha:evaluation}} explains how we set up the experiments and
thoroughly analyzes  the evaluation results of our models.
\\
\\
\textbf{Chapter \ref{cha:conclusion}} summarizes the thesis, describes the problems that
occurred and gives an outlook about future work.
