\thispagestyle{empty}
\vspace*{1.0cm}

\begin{center}
    \textbf{Abstract}
\end{center}

\vspace*{0.5cm}

\noindent

\acrfull{gan}s have shown a great potential for being a powerful implicit model. They have
achieved impressive results in image generation and image-to-image translation. The
ability to capture a complex data distribution makes \acrshort{gan} a good candidate for
representation learning and data augmentation, which are effective tools for improving
\acrlong{ml} performance.
\\

In this Master Thesis, we train some models of \acrfull{gan} to augment an 3D image
dataset which are useful in enhancing the robustness of an Object Classifier. One of the
\acrshort{gan}s shows a good ability to translate a normal 2D image to a 3D depth map. The
results are good in terms of both human eyes and quantitative measurements in a baseline
Deep Classifier. Our experiments show that \acrshort{gan} synthesized data is useful when
being added in large quantity to an incomplete (to some extent) training dataset. As Deep
Learning has been the dominant \acrfull{ai} tool in Computer Vision in the last decade,
and training a Deep Neural Network requires a huge training dataset, our approach reveals
a potential direction that can help training a Deep Classifier easier.
\\

In the way around, it is also very hard to evaluate \acrshort{gan}s' data because these
models do not minimize a normal scalar loss function, which means traditional quantitative
methods such as \acrfull{mse} are not able to show how good the synthesized data is. As
\acrshort{gan}s are popular for generating realistic photos, evaluating them by a very
well-trained Object Classifier could be a good approach.
